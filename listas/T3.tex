\documentclass[12pt]{article}
\usepackage[a4paper, left=2cm, right=2cm, top=2cm, bottom=2cm]{geometry}
\usepackage[brazil]{babel}
\usepackage{amsmath}
\usepackage{minted}
\usepackage{caption}
\usepackage{amssymb}
\usepackage{subcaption}
\usepackage{graphicx}

\title{Introdução à Programação em Python\\
\large Trabalho 3: Estudo e Apresentação de Bibliotecas de Código Aberto}
\author{Prof. Evandro C. R. Rosa\\UNIVALI}
\date{}

\begin{document}
\maketitle

\section*{Objetivo}
Neste trabalho, cada grupo de até 3 pessoas deverá escolher uma biblioteca de Python para estudar e apresentar. O Python possui uma vasta gama de bibliotecas de código aberto desenvolvidas pela comunidade, como \texttt{numpy}, \texttt{pandas}, \texttt{matplotlib}, entre muitas outras. Essas bibliotecas facilitam o desenvolvimento em diversas áreas e contextos. A proposta deste trabalho é permitir que cada grupo se aprofunde no uso de uma dessas bibliotecas, apresentando suas funcionalidades, aplicações e exemplificando seu uso prático.

\section*{Requisitos}
\subsection*{Escolha da Biblioteca}
\begin{itemize}
  \item A biblioteca escolhida deve ser de código aberto e possuir uma comunidade ativa de desenvolvedores.
  \item Cada grupo deve escolher uma biblioteca diferente de outro grupo.
  \item A escolha da biblioteca deve ser confirmada previamente com o professor, para garantir que não haja duplicidade de temas entre os grupos.
\end{itemize}

\subsection*{Desenvolvimento de Código}
\begin{itemize}
  \item O grupo deve desenvolver um código que demonstre as principais funcionalidades da biblioteca escolhida.
  \item O código de exemplo deve ser claro, bem documentado e submetido para o material didático.
\end{itemize}

\subsection*{Apresentação Oral}
\begin{itemize}
  \item O grupo deve preparar uma apresentação oral de 10 minutos sobre a biblioteca escolhida.
  \item A apresentação deve abordar os seguintes pontos:
    \begin{itemize}
      \item \textbf{Domínios de Aplicação:} Quais áreas ou problemas a biblioteca visa solucionar.
      \item \textbf{Funcionalidades Principais:} Quais são as principais funcionalidades oferecidas pela biblioteca.
      \item \textbf{Instalação:} Como instalar a biblioteca (exemplo de comando de instalação).
      \item \textbf{Exemplos Básicos de Uso:} Demonstração de como utilizar a biblioteca com exemplos simples e claros.
    \end{itemize}
  \item Caso a biblioteca tenha uma vasta gama de possibilidades, o grupo deve escolher uma aplicação específica para apresentar, focando nas funcionalidades mais relevantes e práticas.
\end{itemize}

\section*{Instruções Finais}
\begin{itemize}
  \item Não é necessário entregar o material da apresentação (slides), apenas o código de exemplo desenvolvido.
  \item Certifique-se de que o código enviado seja funcional e esteja acompanhado de comentários explicativos.
\end{itemize}

\end{document}
